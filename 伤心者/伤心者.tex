%-*- coding: UTF-8 -*-
% gougu.tex
% 勾股定理


\documentclass[UTF8]{ctexart}
\usepackage{graphicx}
\usepackage{float}

\usepackage{amsmath}
\usepackage{cite}
\usepackage{geometry}
\usepackage{url}
\newcommand{\upcite}[1]{\textsuperscript{\textsuperscript{\cite{#1}}}}
\title{伤心者}
\author{何夕}
\date{2003年}
\zihao{-4}\linespread{1.5}\selectfont
\begin{document}\large

\maketitle
\section*{第一章}
    上午的菜场正是最繁忙的时候,我看着夏群芳穿过拥护的人群-她的背影很臃肿。隔着两三米的距离我看不清她买了些什么菜,不过她跟小贩们的讨价还价声倒是以听得很清楚。从这两天的经历我知道小贩们对夏群芳说话是不太客气的,有时候甚至于就是直接的奚落。不过我从未见过夏群芳为此而表现出生气什么的,她似乎只关心最后的结果,也就是说菜要买得合算,至于另的事情至少从表面上看去她是不计较的。现在她已经买完菜准备离开,我知道她要去哪儿。
    
    这座城市的四月是最漂亮的时候,各个角落里都盛开着各种各样的花。气候不冷也不太热,老年人皮帽还没取小姑娘们就钻空在天气晴朗的时候迫不及待地穿起了短裙,这本来就是乱穿衣的时候呢。"乱花渐欲迷人眼"在这样的季节里成了不折不扣的双关说话。
    
    夏群芳对街景显然并没有欣赏的打算,她只是低着头很费劲地朝公共汽车站的方向走,装满蔬菜的篮子不时和她短胖的小腿撞在一起,使得她每走几步就会有些滑稽地打个趔趄。道路两旁的行道树都是清一色的塔松,在这座温带城市里这种树比原产地要长得快,但木质也相对要差一些。夏群芳今天走的路线与平时稍有不同,因为今天是星期天,她总是在这个时候到C大去看她的儿子何夕。
    
    由于历史的原因,C大的校园网被一条街道分成了两个部分,在这条街上还开着一路公共汽车。夏群芳下车后进入校园的东区,现在是上午十点,她直接朝着图书馆的方向走去,她知道这个时候何夕肯定在那里。同样由于历史的原因,C大的图书馆有两个,分别位于东西两个区。实际上C大的东西两区曾经是两所独立的高校,用校方的语言来说这两所学校是合并,但现在的校名沿用了东区的,所以当年从西区那所学校毕业的不少学生常常戏称自己是亡校奴并只对西区的那所学校寄予母校的情怀。何夕严格来讲也该算是亡校奴,不过何夕是在合并后才开始攻读C大的硕士学位,所以在何夕心中母校就是东区和西区的整体。
    
    何夕坐在东区图书馆底楼的一个角落里悄悄地注视着他,窗外的人就是何夕的母亲夏群芳,她饶有兴趣看着聚精会神的何夕,汗津津的脸上荡漾着止不住的笑意。我看得出她有几次都想拍打窗户打个招呼,但她伸出手却最终犹豫了。
    
    倒是临近窗户坐着的两个漂亮女生发现了窗外的夏群芳,她们有些讨嫌地白了她几眼。
    
    夏群芳看懂了她们的这种眼神,不过好心情好不和她们计较,她有个读硕士的儿子呢,夏群芳在单位里可风光了。想到单位,夏群芳的心情变得有些差,她已经四个月没有从那个单位拿到钱了。当然她四个月并没有去上班,她下岗了,现在摆着个杂货铺,按照夏群芳一向认为合理的按劳取酬的原则,她觉得这也是很自然的事情。夏群芳在窗外按惯例站了二十来分钟,她的表情显得心满意足。我算了一下,为了这一语不发的二十分钟夏群芳提着十来斤东西多绕了五公里路,这种举动虽然不是经济学家的合理行为,但是却是夏群芳的合理行为。
    
    其实今天夏群芳是最没有理由来看何夕的,因为今天是星期天,何夕虽然住校但是星期天总是会回家一趟。不过他不会在家里住,吃过晚饭又不会回学校。何夕知道在何夕的心里学校比家好,不过对于这一点夏群芳并不在意,只是儿子觉得高兴她也就高兴。夏群芳永远都不会知道此刻摊放在何夕面前的那部大部头里有什么吸引人的东西,但很肯定的是每当夏群芳看到儿子聚精会神地沉浸在书中的时候她的心里就一种没来由的欣慰感。这种感觉差不多在何夕刚小学的时候就成型了。她以前就从探究何夕读的是本什么书,更不用说现在何夕读的那些英文原著。从小到大何夕在学业上的事情都是自己做入,甚至包括考大学填志愿选专业,以及当后来大学毕业时由于就业形势不好又转回去读硕士时等等都是如此。想起儿子前年毕业时四处奔波求职时的情形,夏群芳就感到这个世界变化得实在太快,她从没有想到过大学生也有难找工作的一天,在夏群芳的心里这简直无异于天方夜谭。有个同事对夏群芳说这算啥,人家发达国家早就有这种事情了,说话的时候那人脸上有幸灾乐祸的神情。不过事实却肯定地告诉夏群芳的确没有一个好单位肯要她心中无比优秀的儿子何夕,她隐约地听说这似乎和何夕的专业不好有关。不过在夏群芳看来何夕的专业蛮好的,好象叫做什么什么数学。在夏群芳看来这个专业是挺有用的,哪个地方都少不了要写写算算,写写算算可不就是什么什么数学嘛。夏群芳有一次忍不住把自己的想法讲给何夕听,但何夕只是淡淡地笑了一下。夏群芳的心中早就有了主见,自己的儿子可没有什么不好,儿子的专业也是顶好,那些不会用人的单位是有眼无珠,迟早要后悔死的。夏群芳有时候没事就在相有一天等何夕读完硕士后找个好工作一定要气气当初那些不识好歹的人,想到得意处便笑出声来。夏群芳有些不舍地又回头看了眼专心看书的儿子,然后才满怀踏实地欣欣然离去了。
\clearpage
\section*{第二章}

何夕抬起头来,向着我站的方向看过来。我愣了一下,立刻醒悔到他是在看夏群芳的背影。这里坐在窗边的那两个女生开始议论说刚才那个在外边傻乎乎看了半天的人不知是谁,何夕有些恼怒地瞪了她们一眼。他其实很早就知道母亲站在窗户外注视着自己,在他的记忆里母亲几乎每个星期天的上午都会到学校的图书馆来看自己看读书。何夕知道母亲之所以选在这一天来纯粹是前几年的习惯所致,实际上母亲现在的每一天都可以说是假日,因为她下岗了。何夕看着母亲远去的背影叹了口气,他觉得自己的情形也差不了多少。有时候何夕的心里会隐隐地升起一股对母亲埋怨,他觉得母亲实在太将就自己了。从小到大的许多事情她几乎都由何夕自己做主,如果当初母亲能够在选择专业上不要过分顺从自己就好了。何夕摇摇头,觉得自己不该这样埋怨母亲,他其实知道母亲并不是不想帮自己,而是实在没有这方面的见识。

何夕看了下表,急促地向窗外扫视了一下。按理说江雪应该来了,他们说好上午十一点在图书馆里碰面的。何夕简单收拾了一下朝外面走去,刚到门口里就看到了江雪。

和何夕比起来江雪应该算是现代青年了,单从衣着上讲江雪就比何夕领先了五年。这样讲好象不太准确,应该说是何夕落后了五年,因为江雪的打扮正是眼下最时兴的。发型是一种精心雕琢出来的叫做"随意"的新样式,脑后用丝质手绢绾了个小巧的结,衬出她粉白的面庞益发地清丽动人。看着那条手绢何夕心里感到一阵温暖,那是他送给江雪的第一件礼物。手绢上是一条清澈的江河,天空中飘着洁白的雪花,他觉得这条手娟简直就是为江雪定做的一样。看到他们俩人走在校园里的背影很多人都会以为是一个学生在向老教授请教问题,不过江雪并不觉得这样有什么不妥,尽管要好的几个女生提到何夕时总是开玩笑地问"你的老教授呢"。小时她和大她两岁的何夕是邻居。有过一些想起来很温謦的儿时回忆。后来由于母亲的工作变动而分开了,但却很巧地在十多年后的C大又遇上了。当时江雪碰到了迎面而来的何夕,两人不约而同地喊道"哎,你不就是……哎…

…那个……哎吗",等到想起对方名字后两人都大笑起来,所以后来两人还常常大声地称呼对方为"那个哎"。江雪觉得何夕和自己挺合得来,别人的看法她并不看重,她知道几个计算机系还有高分子材料系的男生在背地里说他们是鲜花和牛粪。在江雪看来何夕并不像外界所认为的那样是一个迂腐的书呆子,恰恰相反,江雪觉得何夕身上充满了灵气。给江雪印象最深的是何夕的眼睛,在此之前她从未见过谁拥有这样一双睿智的眼睛,看到这双眼睛的时候江雪总止不住地想有着这样一双眼睛的人一定是不平凡的。

每当看到江雪的时候何夕的心情就变得好,实际上也只有这时候他才有如释重负的感觉。何夕很小就知道自己的性格缺陷。当他手里边有事情没有完成的时候总是放不下,无论做别的什么事情总还惦记着先前的那件事。他本以为自己这辈子都是这种性格了,但江雪的出现改变了一切。和江雪在一起时他也不知道为什么自己就像换了一个人,那些不高兴的事,那些未完成的事都可以抛在脑后,甚至包括"微连续"。一想到"微连续"何夕不禁有些分神,脑子里开始出现一些很奇特的符号。但也立刻收回了思想,实际上只有在江雪到来时才收回了思想,实际上只有江雪到来时他才会这样做,同时也只有在江雪到来时他才做得到这一点。江雪注意到何夕一刹那的走神,在她的记忆里这是常有的事。有时大家玩得正开心的时候何夕却很奇怪地变得无声无息,眼睛也很飘渺地盯住虚空中的不知什么东西。这种情形一般不会持续很长,过了一会儿何夕会自己"醒"过来,就像从睡梦中醒来一样。这样的情况多了大家也就尖意了,只把这理解成每个人都可能有的怪僻之一。

"先到我家午饭,我爸说要亲自做拿手菜。"江雪兴致很高地提议,"下午我们去滑旱冰,老麦才教了我几个新动作。"何夕没有马上表态,眼前浮现出的是老麦风流倜傥的样子来。老麦是计算机系的硕士研究生,也算是系里的几个大才子之一,当初同位居几大佳人这列的江雪本来就开始有了那么一点意思,但是何夕出现了。用老麦的话来说就是"自己想都想不到会输给了江雪的儿时回忆"。渤老麦却是一个洒脱之人,几天过后便又大大咧咧地开始约江雪玩,当然每次都很君子地邀请何夕一同前往。从这一点讲何夕对老麦是好感多于提防,不过有时候连何夕自己也不得不承认老麦和江雪站一起的时候显得那样协调,无论是身材相貌还是别的,这个发现常常令何夕一连几天都心情黯然。但是江雪的态度却是极其鲜明,她毫不掩饰自己对何夕的感情。有一次老麦有点不屑地说"小孩子的感情靠不住",结果江雪出人意料地激动了,她非要老麦为这句话道歉,否则就和他绝交,结果老麦只得从命。

\clearpage

\section*{第三章}
当时老麦的脸上虽然仍旧挂着笑,但何夕看得出老麦差点儿就扛不住了。在这件事情之后老麦便再也没有作任何形式的"反扑"-如果那算是一次反扑的话。

何夕在犹豫要不要答应江雪,他每个星期天都答应母亲回家吃晚饭的,如果去滑旱冰晚上就赶不到回去吃饭的时间了。但是江雪显然对下午的活动兴致很高,何夕还在考虑的时候江雪已经快乐地接着他朝她家跑去,那是位于学校附近的一套商品房。路上江雪银玲一样美妙的笔声驱跑了何夕心中最后的一丝犹豫。

江北园解下围裙走出厨房,饶有兴致地看着江雪很难称得上娴淑的吃相。退休之后他简直可称为神速地练就了一手烹调手艺,高兴得江雪每次大快朵颐之后都要大放厥词称他本来就不该是计算机系的教授而应当是一名厨师。也许正是江雪的称赞使他终于拒绝了学校的聘请。何夕有些局促地坐在江雪的身旁,半天也难得动一下筷子。江家布置得相当有品味,如果稍作夸张的话可称得上一般性的豪华。以江北园的的眼光来看何夕比以前常来玩的那个叫什么老麦的小伙子要害羞得多,不知道性格活泼的江雪怎么会做出这样的选择。不过江北园知道世上有些事情是不能够讲道理的,女儿已经长大了,家里人已经不能像以前那样代她去作判断了。

"听小雪说你是数学系的硕士研究生。"江北园问道。

何夕点点头:"我的导师是L。""L。"江北园念叨着这个名字,过了一会有些不自然地笑笑说,"退休后我的记忆不如以前了。"何夕有脸微微发红:"我们系的老师都不太有名,不像别的系。以前我们出去时提起他们的名字很多人都不熟悉,所以后来我们都不提了。"江北园点点关,何夕说的是实情。现在C大最有名的教授都是诸如计算机系外语系电力系的,不仅是本校,就连外校和外单位的人都知道他们的大名-有些是读他们的编写的书,有的是使用他们开发的应用系统。不久前C大出了件闹得沸沸扬扬的事情,一位学生发明的皮革鞣制专利技术被一家企业以七百万花买走,而后皮革系的教授们也荣升这一行列。

"你什么时候毕业。"江北园问得很仔细。

"明年春季。"何夕慢吞吞地挟了一口菜,感觉并不像江雪说的那样好吃。

"联系到工作没有。"江北园没有理会江雪不满的目光,"已经没有多少时间了。"何夕的额头渗出了细小的汗珠,他觉得嘴里的饭菜都味同嚼腊。"现在还没有。我正在找,有两家研究所同我谈过。另外刘教授也问过我愿不愿意留校。"江北园沉吟了半晌,老实说何夕的回答只是让他放心但并没有让他欢心。他转头看着笑咪咪的女儿,她正一眼不眨地盯着何夕看,仿佛在做研究。

"你有没有选修其它系的课程?"江北园接着问。

"老爸,"江雪生气地大叫,"你要查户口吗?又不是你同何夕谈恋爱,问那么多干嘛?"

江北园立时打住,过了一会儿说:"我去烧汤。"汤端来了,冒着热氯。没有人说话,包括我。

老麦姿态优美地滑过一圈弧线,劝作如行云流水般酣畅。何夕有些无奈地看着自己脚下凭空多出来的几只轮子,心知自己决不是这块料。江雪本来一手牵着何夕一手牵着老麦,但几步下来便不得不放开了何夕的手-除非她愿意陪着何夕练摔筋斗的技巧。

这是一家校外叫做"尖叫"的旱冰场,以前是当地科协的讲演厅,现今承包给个人改装成了娱乐场。条件比在学校里的要好许多,当然价格是与条件成正比的。由于跌得有些怕了,何夕便没有上场,而是斜靠着圈栏很有闲情般地注视着场内嬉戏的人群。当然,他目光的焦点是江雪。老麦正在和江雪练习一个有点难度的新动作,他们在场里穿梭往来的时候就像是两条在水中翩游弋的鱼。这个联想让何夕有些不快。

江雪可能玩得累了,她边招手边朝何夕滑过来,到眼前时却又突然打了一个三百六十度的急旋方才稳稳停住。老麦也跟着过来,同时举手向着场边的小摊贩很潇洒地打着响指。于是那个矮个子服务生忙不迭地递过来几听饮料,老麦看着牌子满意地笑着说你小子还算有点记性。
\clearpage

\section*{第四章}
江雪一边擦汗一边啜着饮料,不时仰起神采飞扬地同老麦扯几句溜冰时的趣事。你撞着那边穿绿衣服的女孩好几次,江雪指着老麦的鼻尖大声地笑着说,别不承认,你肯定是有意的。老麦满脸无辜地摇头,一副打死也不招的招势,同时求救地望着何夕。何夕觉得自己在这个问题上帮不了老麦,只好装糊涂地看着一边。算啦,江雪笑嘻嘻地摆摆手,我们放过你也行,不过今天你得买单。老麦如释重负地抹抹汗说,好啦,算我蚀财免灾。何夕有点尴尬地看着老麦从兜里掏出钱来,虽然大家是朋友,但他无法从江雪那种女孩子的角度反这看作一件理所当然的事,至少有一点,他觉得他总是由老麦做东是一件令他难以释怀的事。但想归想,何夕也知道自己是无力负担这笔开支的。老麦家里其实也没有给他多少生活费,但是他的导师总能揽到不少活,有些是学校的课题,但更多的是帮外面的单位做系统。比方说一些小型的自动控制,或是一些有关模式识别方面的东西,以及帮人做网页,甚至有些根本就是组一个简单的计算机局域网,虽然名称叫做什么综合布线。这所名校的声誉给他们招来了众多客户,在老麦看来他们都是些对高校充满盲目迷恋的外行。很多时候老麦要同时开几处工,虽然他所得的只是导师的零头,但是已足够让他的经济水准在学生中居于上层了,不仅超过何夕,而且肯定也超过了何夕的导师刘青。在何夕的记忆里除了学校组织的课题之外他从未接到过别的工作,何夕有一次闲来无事的时候把自己几年参与课题所得加总在一起之后发现居然还差一块钱才到一千元。接下来的几小时里何夕简直动破了脑筋想要找出自己可能忽略了收以便能凑个整数,但直到他启用了当代数学最前沿的算法也没能再找出哪怕是一分钱。

"今天玩得真高兴。"江雪意犹未尽地擦拭着额上的汗水。老麦正在远处收费处结帐,不时和人争论几句。何夕默不作声地脱着脚上的旱冰鞋,他这才感到这双脚现在又重新属于自己了。

"四点半不到,时间还早呢。"江雪看表,"要不我们到‘金道‘保龄球馆去。"何夕迟疑了片刻:"我看还是在学校里找个地方玩吧。"江雪摆头,乌黑的长发掀起了起伏的波浪:"学校里没有什么好玩的,都是些老花样。还是出去好,反正有老麦开钱。"何夕的脸突然涨红了:"我觉得老让别人付钱不好。"江雪诧异地盯着何夕看:"什么别人别人的,老麦又不是外人。他从来不计较这些的。"

"他不计较可我计较。"何夕突然提高了声音。

江雪一怔,仿佛明白了何夕的心思。她咬住嘴唇,有些不知所措地看着四周。这时老麦兴冲冲地跑回来,眼前的场面让他有些出乎意料。"怎么啦?"老麦笑嘻嘻地问"你们俩在生谁的气?"他看看表,"现在回去太早啦,我们到‘金道‘去打保龄球怎么样?"何夕悚然一惊,老麦无意中的这句话让他心里发冷。又是"金道",怎么会这么巧,简直就像是-心有灵犀。他看着江雪不想正与她的目光撞个正着,对方显然明白了他的内心所想-她真是太了解他了,江雪若有所诉的眼光像是在告白。

"算了。"何夕叹了口气,"我今天很累了,你们去吧。"说完他转身朝室外走去。

江雪倔强地站在原地不动,眼里滚动着泪水。

"我去叫他回来。"老麦说着话转身欲走。

"不用了。"江雪大声说,"我们去‘金道‘。"我正意识地挡在何夕的面前,但是他笔直地朝我压过来并毫无阻碍地穿过了我的身躯。

十八英寸电视里正放着夏群芳一直看着的一部电视连续剧,但是她除了感到那些小人儿晃来晃去之外看不出别的。桌上的饭菜已经热了两次,只有粉丝汤还在冒着微弱的热气。夏群芳忍不住又朝黑漆漆的窗外张望了一下。

有电话就好了,夏群芳想,她不无紧张地盘算着。现在安电话是便宜多了,但还是要几百块钱初装费,如果不收这个费就好了。夏群芳想不出何夕为什么这么晚没有回来吃饭,在印象中这是从来没有的事情。何夕只要答应她的事情从来都是作数的,哪怕只是像回家吃饭这样的小事,这是他们母子多年来的默契。夏群芳又看了眼桌上的饭菜,她没有一点食欲,但是靠近心口的地方却隐隐地有些痛起来。夏群芳撑起身,拿瓢舀了点粉丝汤,而就在这个时候门锁突然响了。
\clearpage

\section*{第五章}
"妈。"何夕推着门就先叫了声,其实这时他的视线还被门挡着,这只是许多年的老习惯。

夏群芳从凳子上站起来,由于动作太急凳子被碰翻在地。"怎么这么晚才回来?"虽然是责备的意思但是她的语气却只有欣喜了,"饿了吧,我给你盛饭。"何夕摆摆手:"我在街上吃过了,有同学请。"夏群芳不高兴了,"叫你少在街上乱吃东西的,现在流行病很多,还是学校里的干净。你看对门家的老二就是在外不注意染上肝炎的……"夏群芳自顾自地念叨着,她没有注意到何夕有些心不在焉。

"我知道啦。"何夕打断她的话,"我回来拿衣服,还要回学校去。"夏群芳这才注意到何夕的脸有些发红,像是喝了点酒,她有些不放心地问:"今天就不回学校了吧?都八点钟了。"何夕环视着这套陈设简陋的两居室,有好一会儿都没有出声。"晚上刘教授找我有事。"他低声说,你帮我拿衣服吧。"夏群芳不再有话,她转身进了里屋,过了几分钟拿着一个撑得鼓鼓的尼龙包出来。何夕检视了一下,朝外拎出几件厚毛衣:"都什么时候了还穿得住这些。"夏群芳大急,又一件件朝口袋里塞:"带上带上,怕有倒春寒呢。"何夕不依地又朝外拎,他有些不耐烦:"带多了我没地方放。"夏群芳万分紧张地看着何夕把毛衣统统扔了出来,她拿起其中一件最厚的说:"带一件吧,就带一件。"何夕无奈地放开口袋,夏群芳立刻手脚麻利地朝里面塞进那件毛衣,同时还做贼般地往里面多加了一件稍薄的。

"怎么没把脏衣服拿回来。"夏群芳突然想起何夕是空手回来的。

"我自己洗了。"何夕转身欲走。

"你洗不干净的。"夏群芳嘱咐道,"下次你还是拿回来洗,你读书已经够累了。再说你干不来这些事情的。""噢。"何夕边走边懒懒地答应着。

"别忙,"夏群芳突然有大发现似地叫了声,"你喝口汤再走。喝了酒之后是该喝点热汤的。"她用手试了一下温度,"已经有点冷了,你等几分钟我去热一下。"说完她端起碗朝厨房走去。等她重新端着碗出来时却发现屋子里已经空了。

"何夕。"她低声唤了声,然后目光便急速地搜寻着屋子,她没有见到那两件已经塞进包里的毛衣,这个发现令她略感放心。这里一阵突如其来的灼痛从手上传来,装着粉丝的碗掉落在地上发出了清脆的响声。

夏群芳吹着手,露出痛楚的表情,这使得她眼角的皱纹显得更深。然后她进厨房里拿拖把。

我站在饭桌旁,看着地上四处横流的粉丝汤,心里在想这个汤肯定好喝至极,胜过世上所有的美味珍稀。

刘青关上门,象征性地隔绝了小客厅里的嘈杂,在这种老式单元房里的声音是可以四处周游的。学校的教师宿舍就这个条件,尤其是数学系,不过还算过得去吧。

何夕坐在书桌前,刚才刘青的一番话让他有些茫然。书桌上放着一叠足有五十厘米高的手稿,何夕不时伸出手去翻几页,但看得出他根本心不在焉。

"我已经尽力了。"刘青坐下来说,他无不爱怜地看着自己最得意的学生。

"我为了证明它花费了十年时间。"何夕注视着手稿,封面上是几个大字-微连续原本,"所有最细小的地方我都考虑到了,整个理论现在都是自治的,没有任何矛盾的地方。"何夕咽了口唾沫,喉结滚动了一下,"它是正确的,我保证,每一个定理我都反复推敲过多次,它是正确的。现在只差最后的一个定理还有些意义不明确,我正试图用别的已经证明过的定理来代替它。"刘青微微叹口气,看着已经有些神思恍惚的何夕:"听老师的话,把它放一放吧。""它是正确的。"何夕神经质地重复着。

"我知道这一点。刘青说,"你提出的微连续理论及大概的证明我都看过了,以我的水平还没有发现有矛盾的地方,证明的过程也相当出色,充满智慧。说实话,我感到佩服。

"刘青回想着手搞里的精彩之处,神情不禁有些飞扬-无论如何这是出自他的学生之手,有一句话刘青没有说出来,那就是他并没有完全看懂手搞。许多地方作的变换式令他迷惑,还有不少新的要领的东西也让他接受起来相当困难。换言之,何夕提出的微连续理论似乎是一套全新的东西,它不能归入以往的任何一个体系里去。
\clearpage

\section*{第六章}

问题是,"刘青小心地开口,他注视着何夕的反应,"我不知道它能用来干什么。"何夕的脸上立刻变得发白,他像是被什么重物击中了一般,整个人都蔫了一头。过了半晌他才回过神来强调说:"它是正确的,我保证。"他仿佛只会说这一句话了。

"我们的研究终究要获得应用才是有意义的,否则只能误入为数学而数学的歧途。""可它看起来是那样的和谐。"何夕争辩道,"充满了既简单又优美的感觉。老师,我记得你说过的,形式上的完美往往意味着理论上的正确。"刘青一怔,他知道自己说过这段话,也知道这段话其实是科学巨匠爱因斯坦的经验之谈。他不否认微连续理论符合这一点,当他浏览着手稿的时候内心的确有种说不出的充满和谐的感受,就像是在听一场完全由天籁之声组成的音乐会。

但问题的症结在于他实在看不出来这套理论会有什么用。自从两个月前何夕第一次向他展示了微连续理论的部分内容后他一直关心这个问题,这段时间他经常从各种途径查找这套理论可能获得应用的范畴,但是他失败了。微连续理论似乎跟所有领域的应用都沾不上边,而且还同主流的数学研究方向背道而驰。刘青承认这或许是一套正确的理论,但却是一套无用的正确理论。就好比对圆周率的研究一样,现在据称已经推算到小数点后几亿位了,而且肯定是正确的,但是这也肯定是无意义的。

"想想中国古代的数学家祖冲之,他只是把圆周率推算到小数点后几位,但他对数学的贡献无疑要比现在那些还在为小数点后几亿位努力的人大得多。"刘青幽幽地说,"因为他做的才是有意义的工作,而不是纯粹的数学游戏。"何夕有些发怔,他听得出刘青话中的意思。"我不同意。"何夕说,"老师,你知不知道,许多年前的某个清晨我突然想到了微连续,它就像是一只无中生有的虫子般钻进了我的脑子。那里它只是一个朦朦胧胧的影子,这么多年来我为了证明它费尽心力。现在我就要完成了,只差最后一点点。"何夕的眼神变得飘渺起来,"也许再有一个月……"刘青在心里轻叹一所,他看得出何夕已经执迷太深。何夕是他所见过的最聪明的数学奇才,按刘青私下的想法,何夕的水平其实可以给这所名校所有的数学教授当老师,他深信只要假以时日何夕必定会是将来数学领域内的一朵奇葩。而现在何夕却误入歧途,陷在了一个总是里,这个情形是刘青忍不住回想起很多年前的自己,那时他也常常因为一些磨人但却无用的数学谜题而废寝忘食形销骨立。但是何夕没有看到问题的关键,刘青知道自己作为师长有义务提醒这一点,尽管这显得很残酷。

"你想过微连续理论可能应用在什么领域吗?我是说,即使作最大胆的想像。"刘青尽量合自己的声音柔和些,虽然他知道这并没有什么用。

何夕全身一震,脸色变得一片苍白。"我不知道。"他说,然后抱住了头。

我看到何夕脚下铺着劣质瓷砖的地面上涸出了一滴水渍。

"这两天我没和江雪在一起。"老麦低声说,坐在桌子对面的他的目光有些躲闪。

何夕有点愤怒地盯着老麦:"你这算是什么意思。江雪和我吵架只是我们两个人的事,你这样做是趁人之危。"老麦啜口茶,眼里升起无奈的神色:"我的确没和江雪在一起。不过我猜想她可能是和老康在一起。""谁是老康?"何夕问,他在脑子里搜索着。

"老康是一家规模不小的计算机公司的老板,那天你和江雪闹别扭之后我们在保龄球馆碰上的。大家是校友,自然谈得多一样。"老麦不无称羡地说,"听说……"他突然打住,目光看向窗外。

何夕回头,江雪从一辆漂亮的宝蓝色小车上下来,她身边一位胖乎乎的年轻人正在锁车。何夕还没想好该怎么办的时候江雪已经很高兴地叫起来:"真巧啊,你们两个也在这儿。"江雪兴奋得满脸发红,她拉着身边的那个人进屋来,对何夕说:"这是康-"她突然一滞,有些发窘地问道,"你叫康什么来着?算啦,我还是叫你老康吧。"然后她指着何夕说,"这是何夕,我的男朋友-"她似乎觉得不够,又补上一句说,"数学系的高材生。""数学系-"老康上下打量着看上去有些猥琐的何夕,伸出手说,"常听小雪提起你。"小雪?何夕心里咯噔了一下,他看了眼江雪,她却是若无其事的样子。"怎么不回我的传呼?"何夕带点气地说。

\clearpage

\section*{第七章}

"让你也急一下。"江雪的表情有些调皮,"谁叫你净气我。好啦,现在让你急了两天,我们俩算是扯平了。今天大家新认识,应该找个地方大吃一顿作为庆祝。我看看,"她煞有介事的盯着三个男人看,然后指着老康说,"我们几个数你最肥,你顿肯定你请啦。"老麦不依地说:"以前请客都是我的专利,这次还是我吧。"老康的表情有些奇怪,他死盯着何夕的脸,仿佛在作某种研究。江雪碰碰他的胳膊:"你干嘛,老盯着何夕看。""我同何夕做不了朋友啦。"老康突然说,语气很是无奈,"我们是情敌,注定要一决高下。""你说什么?"江雪吃了一惊,她的脸立时红了,"何夕是我的男朋友,你不该这样想。"

"我怎么想只有我自己能够决定。"老康咧嘴一笑,目光死死地看着江雪,直到她低下头去。他转头看着何夕说:"我喜欢江雪。"何夕觉得自己的头有点晕,眼前这个胖乎乎的人让他乱了分寸。情敌?这么说他们之间是敌人了,至少人家已经宣战了。何夕感到自己背上已经沁出了汗水,他不知道下一步该做什么,末了他采取了一个也许是最蠢的办法。何夕转头对江雪说:"我该怎么办?"

江雪镇定了些,她正色道:"何夕是我男朋友,我喜欢他。"老康看上去并不意外:"如果你是那种轻易移情别恋的女孩的话我也不会像现在这样喜欢你了。"他举起一只手,服务生跑过来问有什么事。"去替我买九十九朵玫瑰,要最好的。"老康拿出钱。

何夕剧烈地喘着气,他从严没有遇到过这样的事情,这简直就像是戏剧里的情节。"那好吧。"何夕吐出口气,"既然你要和我一决高下的话我一定奉陪。"何夕突然觉得这样的话说起来也是很顺口的,仿佛天生他就最擅长这个。

"我不想待下去了。"江雪说,他的脸依然很红,"我们还是走吧。别人都在看我们。"服务生新送来两杯茶。老康吹了一声短促的口哨,站起身说:"今天的茶我来请。"出乎他的意料的是何夕突然粗暴地将他的手挡开,并且拿出钱说:"谁也不要争,我来。"

何夕默不作声地看着夏群芳忙碌地收拾着饭桌,他不知道自己该怎么开口。

"妈,你能不能帮我借点钱。"何夕突然说,"我要出书。"夏群芳的轻快的动作立时停下来。"借钱?出书?"她缓缓坐到凳子上,过了半晌才问,"你要借多少?""出版社说至少要好几万。"何夕的语气很低,"不过是暂时的,书销出去就能还债的。"

夏群芳沉默地坐着,双手拽着油腻的围裙边用力绞结。过了半晌她走进里屋,一阵"悉悉卒卒"的响动之后她拿着一本存折出来说:"这是厂里买断工龄的钱,说了很久了,半个月前才发下来。一年九百四,我二十七年的工龄就是这个折子。你拿去办事吧。"她想说什么但没有出声,过了一会儿还是忍不住低声补充说,"给人家说说看能不能迟几个月交钱,现在取算活期,可惜了。"何夕接过折子,看了眼金额便朝外走:"人家要先见钱。""等等-"夏群芳突然喊了声。

何夕奇怪地回头问:"什么事?"夏群芳眼巴巴地看着何夕手里那本红皮折子,双手继续绞着围裙的边:"我想再看看总数是多少。""25380,自己做个乘法就行了嘛。"何夕没好气地说,他急着要走。"我晓得了,你走吧。"夏群芳有点不好意思地说,她也觉得自己太罗嗦了。"……

刘青有点忙乱地将桌面上的资料朝旁边抹去,但是何夕还是看到了几个字:研究生入学指南。何夕的眼神让刘青有些讪讪然,他轻声说:"是帮朋友的忙。你先坐吧。"何夕没有落座的意思。"老师。"他低声开口说,"你能不能借点钱给我,我想自己出书。"刘青没有显得意外,似乎早知道会有这事。过了几分钟他走回桌前整理着先前弄乱的资料,脸上露出自嘲的神情:"其实我两年前就在帮人编这种书了。编一章两千块,都署别人的名字。并不是人家不让我署这个名,是我自己不同意,我一直不愿意让你们知道我在做这事。"何夕一声不吭地站着,看不出他在想什么。刘青叹口气说:"我知道你想把微连续理论出书,但是,"他稍顿一下,"没有人会感兴趣的。你收不回一分钱。""那你不打算借钱给我了?"何夕语气平静地问。

\clearpage

\section*{第八章}

刘青摇摇头:"我不愿意眼睁睁地看着你失败。到时候你会莫名奇妙地背上一身债务,再也无法解脱。你还这么年轻,不要为了一件事就把自己陷死在里面。我以前……"门铃突然响了,刘青走出去开门。让何夕想不到的是进门的人他居然认得,那是老康。

老康提着一个漂亮的盒子,看来他是来探访刘青的。刘青正想作介绍,而何夕和老康已经面色凝重地握手了。"原来你们认识。"刘青高兴地搓着手,"这可好。我早有安排你们结识的想法了,在我的学生里你们俩可是最让我得意的。"何夕一怔,他记得老康是计算机公司的老板。老康理解地笑了笑说:"我是数学系毕业的,想不到会这么巧,这么说我算起来还是你的同门师兄。"他促狭地眨眨眼,"怎么样,知道孔融让梨的故事吧。"刘青自然不明白其中的曲折,他兴奋得仿佛年轻了几岁,四下里找杯子泡茶。老康拦住他说不用了,都不是外人。何夕在一旁默默地看着这一切,他看得出这个老康当年必定是刘青深爱的弟子。

"老师。"何夕说,"你有客人来我就不耽搁了。我借钱的事……"刘青脸上的笑容不见了,他盯着何夕的脸,目光里充满惋惜:"你还是听我的话,放弃那些不切实际的想法吧。借钱出这样的理论专著是没有出路的。"他转头对老康解释道:"何夕提出一套新颖的数学理论,他想出书。"老康眼里闪过一个亮点,他插话道:"能不能让我看看,一点点就行。"何夕想了一下,然后从包里拿出几页纸递给老康。老康的目光飞快地在纸页上滑动着,口里念念有词。他的眉头时而紧蹙时而舒展,整个人仿佛沉浸到了那几页纸里。过了半天他才抬起头来,目光有些发呆地看着何夕:"证明很精彩,简直是音乐。"何夕淡淡地笑了,他喜欢老康这样的比喻。其实正是这种仿佛离题万里的比喻才恰恰表明老康是个内行。

"我借钱给你。"老康很干脆地说,"我觉得它是正确的,虽然我并没有看得懂多少。"刘青哑然失笑:"谁也没说它是错的。问题在于这套理论有什么用,你能看出来吗?"老康找头,然后龇了龇嘴,"暂时没看出来。"他紧跟上一句,"但是它看上去很美。"老康突然笑了,因为他无意中说了王朔的小说名,眼下正浒。"不过我说借钱是算数的。

"刘青突然说:"这样,如果你要借钱给何夕必须答应我一条,不准写借据。"何夕惊诧地看着刘青,印象中的老师从来都是温文有礼并且拘泥小节的,不知道这种赖皮话何以从他的口中冒出来。

"那不行。"何夕首先反对。

"非要写的话就把借方写成我的名字,我来签字。如果你们不照着我的话做的话就不要叫我老师了。"刘青的话已经没有了商量的作地。

在场的人只有我不吃尺,因为我知道会发生什么样的事情。

江雪默不吭声地盯着脚底的碎石路面,她不知道何夕会作出什么样的反应。从内讲如果何夕发一通脾气的话她倒还好受一些,但她最怕的是何夕像现在这样一言不发。

"你说话呀,"江雪忍不住说,"如果你真的反对的话我就不出去了。很多人没有出去也干出了事业。"何夕幽幽地开口:"老康又出钱又给你找担保人,他为你好,我又怎能不为你着想。""钱算是我借他的,以后我们一起还。"江雪坚决地说,"我只当他是普通朋友。""我知道你的心意。"何夕爱怜地抚着江雪的脸。

"等我出去站稳了脚你就来找我。"江雪憧憬地笑,"你知不知道,你是我见过的最聪明透顶的人。如果你是学我们这种专业的话早就成功立业了。我说是的真的。"江雪孩子式地强调,"你有这个实力。我觉得你比老康强得多。"何夕心里滑过一丝柔情:"问题是我喜欢我的专业。在我看来那些符号都是我的朋友,是那种仿佛已经认识了几辈子的感觉。只有见到它们我的心里才感到踏实,尽管它们不能带给我什么,甚至还让我吃苦头,但是我内心里有一个声音告诉我,这就是我降临到世上应该做的事情。"江雪调皮地刮脸:"好大的口气,你是不是还想说天将降大任于斯人也……"何夕叹口气:"我的意思是……"他甩甩头,"我入迷了,完全陷进去了。现在我只想着微连续,只想着出书的事。为了它我什么都顾不上了。就这个意思。"江雪不笑了,她有些不安地看着何夕的眼睛:"别这么说,我有些害怕。"何夕的眼睛在月光下闪过莹莹的亮点:"说实话我也害怕。我不知道明天究竟会怎样,不知道微连续会带给我什么样的命运。不过,我已经顾不上考虑这些了。"江雪全身一颤:"你不要用这种口气对我说话好吗,这让我觉得失去了依靠。"失去依靠?何夕有些分神,他有不好的预感。"别这样。"他揽住江雪的肩,"我们现在不是还好好的嘛。无论如何,"他深深地凝视着江雪姣好的面宠,"我永远都喜欢你。"江雪感受到了何夕温热的气息扑面而来,月色之中她柔软的唇像河蚌一样的微微翕开,漫天谜一样的星光下她的眼睛里充满泪水。


\clearpage


\section*{第九章}

这是个错误。我轻声说,但是热吻中的人儿听不到我的话。

"我说服不了他们。"刘青不无歉疚地看着何夕失望的眼睛,"校方不同意将微连续理论列为攻关课题,原因是-"他犹豫地开口,"没有人认为这是有用的东西。你知道的,学校的经费很紧张,所以出书的事……"何夕没有出声,刘青的话他多少有所预料。现在他最后的一点期望已经没有了,剩下的只有自费出书这一条路了。何夕下意识地摸了一下口袋里的存折,那里母亲二十七年的工龄,从青春到白发,母亲连问都没有问一句就给他了。何夕突然有点犹豫,他不知道自己究竟有什么权力来支配母亲二十七年的年华-虽然他当初是毫不在乎地从母亲手里接过了它。

"听老师的话。"刘青补上一句,"放弃这个无用的想法吧。还有很多有意义的事情值得去做,以你的资质一定大有作为的。"出乎刘青意料的是何夕突然失去了控制,他大笑起来,笑出了眼泪。"大有作为……难道你也打算让我编写什么研究生入学考试指南吗?那可是最有用的东西,一本书随便印上几万本,可以让我出名,可以让我赚大笔钱。"何夕逼视着刘青,他的目光里充满无奈,"也许你愿意这样可我没法让自己去做这样的事情。我不管您会怎么想,可我要说的是,我不屑于做那种事。"何夕的眼神变得有些狂妄,"微连续耗费了我十年的时光,我一定要完成它。是的,我现在很穷,我的女朋友出国深造的钱居然用的是另一个男人的钱。

"何夕脸上的泪水滴到了稿纸上,"可我要说的是,没有什么力量能够阻止我。我只知道一点,微连续理论必须由我来完成,它是正确的,这是我的心血。"他有些放肆地盯着刘青,"我只知道这才是我要做的事情。"刘青没有说话,表情有些GANGA,何夕的讽刺让他没法再谈下去。"好吧。"刘青无奈地说,"你有你的选择,我无法强求你,不过我只想说一句-人是必须面对现实的。"何夕突然笑了,竟然有决绝的意味。"还记得当年你第一次给我们讲课时说的第一句话吗?"何夕的眼神变得有些飘渺,"当时你说探索意味着寂寞。那是差不多七年前的事情了,这么多年来我一直都记着这句话。"刘青费力地回想着,他不记得自己说过这话了,有很多话只是在某个场合随便说说罢了。但是他知道自己一定是说过这句话的,因为他深知何夕的记忆力非凡。七年,不算短的时间,难道自己真的已经改变?""问题在于-"刘青试图作最后的努力,"微连续不是一个有用的成果,它只是一个纯粹的数学游戏。""我知道这一点。是的,我承认它的的确确没有任何用处,老实说我比任何人都清楚这一点。"何夕平静但是悲怆地说,这是他第一次这样直接说出这句话。何夕没想到自己能够这样平静地表述这层意思,他以为根本是做不到的事情。一时间他感到心里似乎有什么东西在一点一点地破碎掉,碎成碴子,碎成灰尘。但他的脸上依然如水一样的平静。

"可我必须完成它。"何夕最后说了一句,"这是我的宿命。"

这段时间何夕一直过着一种挥金如土的日子。他的身上从来没有像现在这般阔气,往往随手一摸就是厚厚的一叠钞票。尽管从衣着上他还和以前一样寒酸,加上满脸的胡须,看上去显得老了一头。何夕每日里都匆匆地赶着路,神情焦灼而迫切,整个人都像是被某种预期的幸福包裹着。如果留意他的眼神的话会发现不少有意思的东西,他仿佛变了一个人。如果要给这种眼神找一个准确的描述是相当难的,不过要近似地描述一下还是可以办到的-见过赌徒在走向牌桌时的眼神吗?就是那样,而且还是一个兜里每一分钱都是借来的那种赌徒。

何夕正和一个胖敦敦的眼镜大声争吵,他的脸涨得通红。"凭什么要我交这么多。"何夕不依地问,"我知道行情。"他笨拙地抽烟,尽量显出深于世故的样子。

胖眼镜倒是不紧不忙,这种事他有经验:"你的书的稿里有很多自创的符号,我们必须专门处理,这自然要加大出版成本。要不你就换成常用的。

"那不成。"何夕往皱巴巴的西服袖子上擦着汗,但是他已经没法像刚才那样大声了,"这些符号都是有特殊意义的,是我专门设计的,一个也不能换。微连续是新理论,等到它获得承认之后那些符号就会成为标准化的东西。"胖眼镜稍稍地撇了下嘴,脸上仍然是职业化的笑容。"你说得很对。问题是咱们不赶在标准的前面了嘛,那些符号增大了我们的成本。"他收住笑容,拿出一页纸来,"就这个数,少一分也不行。你同意就签字。"何夕怔怔地看着那张纸,那个数字后面长串的零就像是一张张大嘴,它们扭曲着向何夕扑过来,不断变化着形状,一会儿像是江雪的漂亮的眼睛,一会儿像是刘青无奈的目光。更多的时候就像是老康白白胖胖的笑脸。何夕已经记不清自己向K开了几次口了,每当胖眼镜找出理由抬价的时候他只能去找老康。老康是爽快而大方的,但他白胖的笑脸每次都让何夕有种如芒在背般的感觉受。老康总是一边掏钱一边很豪放的说有什么困难只管开口,你是小雪的朋友嘛。小雪每次来信都叫我帮你,小雪安排的事情要是办不好,等我以后到了那边可怎么交待哟。

\clearpage

\section*{第十章}
何夕面色灰白地掏出笔,他仿佛听到有个细弱的声音在阻止他下步的行动,听上去有些像是江雪。但是他终究在那张纸上签了名,也就在这个时候他内心的那个小声音突然消失了,再也听不见了。

胖眼镜一等到何夕的背影转过楼梯口便露出了得意的笑容,他小心翼翼地收好有何夕答名的那张约会。"雏儿。"胖眼镜不屑地转身,随手将另几页纸扔进了垃圾桶。

我看着那几页纸,它们同何夕签字的那张纸的内容完全一样,只是在填写金额的地方填着另外的数字。那些金额都更小。

"……六月的大湖区就像是天堂。绿得发亮的是草地上是自在的人们。狗和小孩嬉戏着,空气清新得像是能刺透你的费。这里的风景越好越让我想起你。亲爱的,你什么时候来到我身边。我想你。""……老康昨天才走,他出来参加一个秋季产品展示会。难为他从西岸赶到东岸来看我。

在这里能够见到老朋友真是愉快的事,尤其是能新耳从朋友口里听到关于你的事情。我让老康多帮帮你,你也不要见外,朋友间相互帮忙是常有的。其实老康人挺不错的,就是说话比较直一点。""……今天这里下了冬天的第一场雪,我特意和几个朋友赶到了郊外照相。大雪覆盖下的原野变得和故乡没有什么不同,于是我们几个都哭了。亲爱的夕,你真的沉迷在那个问题里了吗?难道你忘了还有一个我吗?老康说你整日只想着看书,什么也不管了。他劝你也不听。你知道吗,其实是我求老康多劝劝你的。听我的话,忘掉那个古怪的问题吧,以你的才智完全还有另外一条铺着鲜花的坦途可走,而我就在坦途的这头等你。听我的话,多为我们考虑一下吧。让我来安排一切。""亲爱的夕,有人说在月色下女人的心思会变得难以捉摸。我觉得这这人说得真好。今夜正好有很好的月光,而我就站在月光下的小花园里。老康在屋里和几个朋友听音乐(他又出来参加什么展示会了),我不知道是不是他有意选择了这首曲子,真是像极了我现在的心情。那些缠绵,带着无法摆脱的忧伤,还有孤独。是的,孤独,此时此刻我真想有人陪着我,听我说话,注视着我,也让我能够注视他。亲爱的夕,我不知道你为什么拒绝我为你安排的一切,难道那个问题真的比我更重要吗?拿出我的像片来看看,看着我的眼睛,它会使你改变的,相信我……老康在叫我了,他总是很仔细,不放心我一个人出来。""……今天和室友吵了一驾,我真是没用,哭得惨兮兮的。也许是一个人在外久了我变得很脆弱,一点小事就想不开。我真想有个坚强的臂膀能够依靠。你离得那么远,就像是在天边。老康下午突然来了(他现在成了展示会专业户了),见我一直哭他就编笑话给我听,全是我以前听过的,要是在以前我早就要奚落他几句了,可这次不知怎么却笑得像个傻孩子。老康也陪着我笑,样子更傻……""……回想当日的一切就像是在做梦,我们有过那么多欢乐的时光。我真的不知道自己究竟应该怎么做。我不是善变的人,直到今天我还这么想。我曾经深信真爱无敌,可我现在才知道这个世界真正无敌的东西只有一样,那就是时间。痛苦也好喜悦也好,爱也好恨也好,在时间面前它们都有是可以被战胜的,即使当初你以为它们将一生难忘。在时间面前没有什么敢称永恒。当我写下这段文字的时候我的泪水止不住地往下流,但这并非因为对你的爱,而是我在恨自己为何改变了对你的爱-我要以为那是不可能的事。老康已经办妥了手续,他放弃了国内的事业。他要来陪着我。就让我相信这是时间的力量吧,这会让我平静。"

夏群芳擦着汗,不时回头看一眼车后满满当当的几十捆书。每本书都比砖头还厚,而且每册书还分上中下三卷,敦敦实实让她生出了满腔的敬畏来。这使得夏群芳想起了四十多年前自己刚发蒙时面对课本时的感觉,当时她小小的心里对于编写出课本的人简直敬若天人。想想看,那么多人都看同一本书,老师也凭着这个来考试号卷打分。书就是标准就是世上最了不得的东西,而写书的人当然就更了不得了,而现在这些书全是她的儿子写出来的。
\clearpage

\section*{第十一章}

在印刷厂装车的时候夏群芳抽出一本书来看,结果她发现自己每一页都只认得不到百分之一的东西。除了少数汉字以外全是夏群芳见所未见的符号,就像是迷信人家在门上贴的桃符。当然夏群芳只是在心里这样想,可没敢说出来。这可是家里最有学问的人花了多少力气才写出来的,哪能是桃符可以比的。让夏群芳感到高兴的是有一页她居然全部看得懂,那就是封面。微连续原本,何夕蓍。深红的底子上配着这么几个字简直好看死了,尤其是自己儿子的名字,原来何夕两个字烫上金这么好看,又气派又显眼。

夏群芳想着便有些得意,这个名字可是她起的。当初和何夕的死鬼老爸为起这名字的事还没有少争过,要是死鬼看到这个烫金的气派名字不服气才怪。

车到了楼下夏群芳变得少有的咋咋呼呼,一会儿提醒司机按喇叭以疏通道路,一会儿亲自探出头去吆喝前边不听喇叭的小孩。好事的邻居全围拢来,不知道发生了什么事。

"买啥好东西了?"有人问。

夏群芳说到了,叫司机停车,下来打开后盖。"我家小夕出的书。"夏群芳像是宣言般地说,她指着一捆捆的敦煌巨蓍,心里简直满得不行,有生以来似乎以今日最为舒心得意。

"哟!"有好事者拿起一本看看封底发出惊叹,"四百块钱一套。十套就是几千一百套就是几万。你家以后怕不是要晒票子了。夏群芳IA阿姨你可要请客哟。"夏群芳觉得自己简直要晕过去了,她的脸发烫,浑身充满了力气。她几乎是凭一个人的力气便把几十捆书搬上了楼,什么肩周炎腰肌劳损之类的病仿佛全好了。这么多收进了屋立刻便显得屋子太小,夏群芳便孜孜不倦地调整着家具的位置,最后把书垒成了方方正正的一座书山,书脊一律朝外,每个人一进门便能看到书名和何夕的烫金名字。夏群芳接下来开始收拾那一堆包装材料,她不时停焉得虎子,偏着头打量那座书山,乐呵呵地笑上一回。

老康站住了,他身后上方是"国际航班通道"的指示牌,身前是大群送行的亲友。何夕和老老麦AI同他道别之后便走到不远之外的一个僻静角落里,与人们拉开了距离。

"我不认为他适合江雪。"老麦小声地说了句,他看着何夕,"我觉得你应该坚持。J是个好女孩。"何夕又灌了口啤酒,他的脸上冒着热气。因为酒精的作用他的眼睛有些发红。

"他是我的同行。"老麦仿佛在自言自语,"我也准备开家电脑公司,过几年我肯定能做到和他一样好。我们这一行是出神话的行业。别以为我是在说梦话,我是认真的。不过有件事我想跟你说说,"老麦声音大了点,"几个月前我认识了一个老外,也是我的同行,很有钱。知道他怎么说吗。他对我说你们太"上面"了。我不清楚他是不是因为中文不好才用了这么一个词,不过我最终听明白了他的意思。他说他并不因为世界首富出在他的国家就感到很得意,实际上他觉得那个人不能代表他的国家。在他的眼里那个人和让他们在全世界大赚其钱的好莱坞以及电脑游戏等产业没有什么本质差别。他说他的国家强大不是在这些方面,这些只是好看的叶子和花,真正让他们强大的是不起眼的树根。可现在的情况是几乎所有的人只盯着那棵巨树上的叶子和花,并徒劳地想长出更漂亮的叶子和花来超过它。这种例子太多了。"何夕带点困惑地看着老麦,他不知道大大咧咧的老麦在说些什么。他想要说几句,但脑子昏昏沉沉的。这些日子以来他时时有这种感觉,他知道面前有人在同自己讲话,但是集中不起精神来听。他转头去看老康,从个子上他并不比老康矮,但是他看着老康的时候感觉自己就像是一个侏儒,须得仰视才行。欠老康多少钱,何夕回想着自己记的帐,但是他根本算不清。老康遵照着刘青的意思不要借据,但何夕却没法不把帐记着。你拿去用。老康胖乎乎的笑脸晃动着,是小雪的意思。小雪求我的事我还能不办啊,啊哈哈哈。烫金的《微连续原本》几个字在何夕眼前跳动,大得像是几座山。每一座就像是家里那座山。几个月了,就像是刘青预见的那样,没有任何人对那本书感兴趣。刘青拿走了一套,塞给他四百块钱,然后一语不发地离开。他的背影走出很远之后让何夕看见轻轻叹口气把书扔进了道旁的垃圾桶。正是刘青的这个举动真正让何夕意识到微连续的确是一个无用的理论-甚至连带回家摆设都不够格。天空里有一本汗津津的存折飞来飞去,夏群芳在说话,这里厂里买断妈二十七年工龄的钱。何夕灌了口啤酒咧嘴傻笑,二十七年,三百六十四个月,九千八百五十五天,母亲的半辈子。但何夕内心里却有一个声音在说,这个世上惟一不用感到内疚的只有母亲。

\clearpage

\section*{第十二章}

书山还在何夕眼前晃动着,不过已经变得有些小了。那天何夕刚到这有夏群芳便很高兴地说有几套书被买走了,是C大的图书馆。夏群芳说话的时候得意地亮着手里的钞票。但是何夕去的时候管理员说篇目上并没有这套书,数学类书架也找不到。何夕说一定有一定有准是没登记上麻烦你再找找。管理员拗不过只得又到书架上去翻,后来果真找出了一套。何夕觉得自己就要晕过去了,他大口呼吸着油墨的清香,又手颤抖着轻轻抚过书的表面,就像是抚摸自己的生命,巨大的小滴掉落在了扉页上。管理员讷闷地嘀咕,这书咋放在文学类里。他抓过书翻开了封面,然后有大发现地说,这不是我们的书,没印章。对啦,准是前天那个闯起来说要找人的疯婆子偷偷塞进去的。管理员恼恨地将书往外面地上一扔,我就说她是个神经病嘛,还以为我们查不出来。何夕简直不知道自己是怎样回到家里的,他仿佛整个人都散了架一般。一进门夏群芳又是满面笑容地指着日渐变小的书山说今天市图书馆又买了两册,还有蜀光中学,还有育英小学。

这里不远处的老康突然打了个喷嚏。国内空气太糟,他大笑着说,然后掏出手帕来擦拭鼻子,手帕上是一条清澈的河流,天空中飘着洁白的雪花。

我伸出手去,想挡住何夕的视线,但是我忘了这根本没有用。

……

"老康打了个喷嚏,"老麦挠挠头说,"然后何夕便疯了。我也不明白是怎么一回事,反正我看到的就是那样。真是邪门。""后来呢。"精神病医生刘苦舟有些期待地盯着神神叨叨的老麦,他觉得此人说不定有望发展成自己的下一个客户。

"何夕冲上去捏老康的鼻子,嘴里说叫你擤叫你擤。他还抢老康的手帕,"老麦苦笑,"抢过来之后他便把脸贴上去翻来覆去地亲。"老麦厌恶地摆头,"上面糊满了NIAN乎乎的鼻涕。之后他便不说话了,一句话也不说,不管别人怎么样都不说。""关于这个人你还知道什么?"刘苦舟开始写病历,语句都是现成的,根本不经过大脑,"我是说比较特别的一些事情。"老康想了想:"他出过一套书。是大部头,很大的大部头。""是写什么的。"刘苦舟来了兴趣,"野史?计算机编程?网络?烹调?经济学?生物工程?或者是建筑学?""都不是。是最老套的东西,数学。""那就对了。"刘苦舟释怀地笑,顺利地在病历上写下结论,"那他算是来对地方了。"这里夏群芳冲了进来,身上还系着油腻的围裙,这使她整个人显得滑JI。她的眼睛红得发肿,目光惊慌而散乱。何夕怎么啦?出什么事啦?好端端的怎么让飞机机撞了?她方寸大乱地问,然后她的视线落到了屋子的左角,何夕安静地坐在那里,眼神飘渺地浮在虚空,仿佛无法对上焦距。他已经不是以前的何夕了,这飘浮的眼光证明了这一点。"让飞机撞了?老麦想着夏群芳的话,他不知道是不是自己在机场报信里说得太快让他听错了。

"医生说治起来会很难。"老麦低声地说。

但是夏群芳并没有听见这句话,她的全部心思已经落到了何夕身上。从看到何夕的时刻她的目光就变了,变得安定而坚定。何夕就在她的面前,她的独生子就在她的面前,他没有被飞机撞,这让她觉得没来由的踏实,她的心情与几分钟之前已经大不一样。何夕不说话了,他紧抿着嘴,关闭了与世界的交往,而且看起来也许以后都不会说话了。不过这有什么关系呢,何夕生下来的时候也不会说话的。在夏群芳眼里何夕现在就像他小时候一样,乘得让人心痛,安静得让人心痛。

(未录入完…。主要情节几乎已经结束,最后还剩下150年后……

我是何宏伟。

一连两天我没有见过一个客人,尽管外界对于此次划时代事件的关注激情已经到了白热化的程序。这两天里我一直在写一份材料。现在我已经写好了。其实这两天我只是写下了几个人的名字,连连同简短的说明。但是每写下一个字我的心里都会滚过长久的浩叹,而当我写下最后那个人的名字时几乎握不住自己的笔。然后我带着这样一份不足半页的资料站到了诺贝尔物理学奖的领奖台上。无论怎么评价我的得奖项目都不会过分,因为我和我的领导的实验室是因为大统一场方程而得奖的。这是人类最伟大的梦想,从某种意义上讲是人类认识的终极。

"女士们先生们。"我环视全场,"大家肯定知道,从爱因斯坦算起为了大统一场理论已经过去了两百多年,至少耗尽了十几代最优秀的人的生命。我是在三十年前开始涉足这个领域的。在差不多十七年前的时候我便已经在物理意义上明晰了大统一理论,但是这时候我遇到无法逾越的障碍。实际上不仅是我,当时有很多人都做到了这一步,但是却再也无法前行一步。你们有过这样的体会吗,就是有一件事情,你自己心里似乎明白了,但却无法把它说出来,甚至根本无法描述它。你张开了嘴,但是却发现吐不了一个字,就像是你的舌头根本不属于你。此后我一直同其他人一样徘徊在神山的脚下,已经看得见事情的转机说来有几分戏剧性。两年前的某末我送十一岁的小儿子去上学,当时他们的一幢老图书楼正被推倒。在废墟里我见到一套装在密封袋里的书,后来我才知道这套书已经出版了一百五十年,但是当时它的包装竟然完好无损,也就是说从未有人留意过它。如果当时我不屑一顾地走开,那么我敢说世界还将在黑暗里摸索一百五十年。但是一股好奇心让我拆开了它,然后你们可以想像我当时的心情,就像是一个穷到极点的乞丐有一天突然发现了阿里巴巴的宝藏。我不知道这样一部我难以用语言来评述的伟大著作怎么会被收藏在一所小学校里,不知道上天为何对我这样好,让我有幸读到这样非凡的思想。我只知道当时我简直失去了控制了,在废墟上大喊大叫不能自己。这正是我要找的东西,它就是大统一理论的数学表达式,甚至比我要的还要多得多。那一时刻我想到了牛顿。他的引力思想并非独有,比如同时代的胡克不能,所以只能是牛顿来解决引力问题。现在我面临的问题又何尝不是这样。书的名字叫《微连续原本》作者叫何夕。

是的,当时我的惊讶并不比你们此刻少。这是个完全陌生的名字。后来的事正如你们看到的,在不到半年的时间里我发表了一系列重要论文,简直是神速地完成了大统一理论的方程式,甚至在几个月前我和我的小组还试制出基于大统一理论的时空转换设备。有人说我是天才,但是今天我只想说一句,超越时代的不是我,而是一百五十年前的那位叫何夕的人。不要以为我这样说会感到难堪,其实我只感到幸运,因为我现在已经知道超越时代意味着什么。如果何夕生在我们的时代根本轮不到我站在这个地方。在他的那个时代支持大统一理论的物理事实少得可怜,现在我们知道必须达到一千万亿吉电子伏特的能级才可能观察到足够多的大统一场物理现象。而在何夕的时代这是不可想象的,这也就注定了他的命运。他是个什么样的人?为何他写下了这样伟大的著作但却被历史的黄沙掩埋?为了解开心中的这些疑团,我将第一次时空实验的时区定在了何夕生活的那个年代。我们安排一个虚拟的观察体出现在了那个过往的年代,那实际上是一处极小的时空洞。它可以随意地出现在指定的时间和地点,从而观察到当时的事情。我亲眼目睹了事情的全部过程,如果诸位不反对的话我想把我知道的全讲出来。"台下没有一个人说话,甚至听不到大声出气的声音。我轻声描述着自己近日来的经历,描述着何夕,描述着何夕的母亲夏群芳,描述着那个时代我见过的每一个人。他们在我的眼前鲜活过来了,连同他们的向往与烦恼。工作人员打开了投影仪,两幅老照片投放在了屏幕上。这是我委托政府找到的,可惜只有两张。一张是年轻漂亮的少妇夏群芳抱着她刚满周岁的胖儿子何夕坐在公园的长椅上,脸上是幸福而憧憬的笑容。别一张是风烛残年的半文盲妇人夏群芳,她拿着一把梳子专注地给她满脸胡须的目光痴呆的傻儿子何夕梳头,目光里充满爱怜。

尽管我想忍住但还是流下了泪水。我觉得照片上的母亲和儿子是那样的亲密,他们都是那样的SHAN良,而同时他们又是那样的-伤心。是的,他们真的很伤心。而现在他们早已离开这个他们一生都无法理解的世界了,就仿佛他们从来没有来过。

"如果没有何夕,大统一理论的完成还将遥遥无期。"我接着说,"而纯粹是由于他母亲的缘故,《微连续原本》才得以保存到今天,当然这燕非她的本意,当初她只是想骗骗自己的儿子,想让他开心。以她的水平根本不知道这里面究竟写的什么东西,根本不知道这是怎样的一本著作,所以她才会将这部闪烁不朽光芒的巨著偷偷放到一所小学校的图书楼里。从局外人的观点看她的行为会觉得荒唐可笑,但她只是在顺应一个母亲的本能。自始至终她只知道一点,那就是她有一个好孩子,这是她的好孩子选择去做的事情。

我不否认对何得心应手那个时代来说《微连续原本》的胡没有什么意义,但我只想说的是,对有一些东西是不应该过多地讲求回报的,你不应该要求它们长出漂亮的叶子和花来,因为它们是根。这是一位母亲教给我的。母亲对自己的孩子永远都不会要求回报,但是请相信我们可爱的孩子自会回报他的母亲。""还有一点,"我稍稍顿了一下,"记得当初在长达几个世纪的时光里有无数人为了永动机耗尽了他们的一生。也许我们可以说这只是一些愚蠢的人,可是正是这些人的探索才最终让我们认识了热力学定律。他们虽然没能告诉后人应当走哪能条路,但却指明了其中的某些路是死路。所以我要说,即使微连续理论在今天仍然被证明是无用的,我们依然应当对何夕表示敬意。因为他曾经尽力求索过,这就够了。"我看着手里的半页纸,上面的每一个名字都是那样的伤心。"也许我们应该永远记住这样一些人。"我照着纸往下念,声音在静悄悄的大厅里回响。

"古希腊几何学家阿波洛尼乌斯总结了圆锥曲线理论,一千八百年后由德国天文学家开普勒将其应用于行星轨道理论。

数学家伽罗华公元1831年创立群论,一百余年后获得物理应用。

公元1860年创立的矩阵理论在六十年后应用量子力学。

数学J。H莱姆伯脱,高斯,黎曼,罗马切夫斯基等人提出并发展了非欧几何。高斯一生都在探索非欧几何的实际应用,但他抱憾而终。非欧几何诞生一百七十年后,这种在当时毫无用处的理论以及由之发展而来的张量分析理论成为爱因斯坦广义相对论的核心基础。

何夕提出并于公元1999年完成的微连续理论,一百五十年后这一成果最终导致了大统一场理论方程式的诞生。"世界沉默着,为了这些伤心的名字,为了这些伤心的名字后面那千百年的寂寞时光。

我拿出一张光盘:"何夕后来一直没有说过话,医生说他已经丧失了语言能力。但是我这里有一段录间,是后来何夕临死前由医院制作为医案的,当时离他的母亲去世不到一个星期。我现在已无法知道这究竟是因为何夕在母亲去世之后失去了支撑呢,还是他虽然疯了但却一直在潜意识里坚持着比母亲活得长久-这也许是非曲直他惟一能够报答母亲的方式了。还是让我们来听听吧。"背景声很嘈杂,很多人在说话。似乎有几位医生在场。放弃吧。一个浑厚的声音说,他没救了,现在是十点零七分,你记下时间。好吧,一个年轻的声音说,我收拾一下。年轻的声音突然升高,听,病人在说话,他在说话。不可能,浑厚的声音说,他已经二十年没说过一句话了,再说话也不可能有力气说话。但是浑厚的声音突然打住,像是有什么发现。周围安静下来,这里可以听见一个带着潮气仿佛已经锈蚀多年来的声音在说着什么。

"妈-妈——"那个声音有些含糊地喊到。

"妈——妈-"他又喊了一声,无比的清晰。


\clearpage
\end{document}